%%%%%%%%%%%%%%%%%%%%%%%%%%%%%%%%%%%%%%%%%
% Beamer Presentation
% LaTeX Template
% Version 1.0 (10/11/12)
%
% This template has been downloaded from:
% http://www.LaTeXTemplates.com
%
% License:
% CC BY-NC-SA 3.0 (http://creativecommons.org/licenses/by-nc-sa/3.0/)
%
%%%%%%%%%%%%%%%%%%%%%%%%%%%%%%%%%%%%%%%%%

%----------------------------------------------------------------------------------------
%	PACKAGES AND THEMES
%----------------------------------------------------------------------------------------


\documentclass[hyperef={
    colorlinks=true,
    linkcolor=blue,
    filecolor=black,
urlcolor=blue}
]{beamer}
\mode<presentation> {

% The Beamer class comes with a number of default slide themes
% which change the colors and layouts of slides. Below this is a list
% of all the themes, uncomment each in turn to see what they look like.

%\usetheme{default}
%\usetheme{AnnArbor}
%\usetheme{Antibes}
%\usetheme{Bergen}
%\usetheme{Berkeley}
%\usetheme{Berlin}
%\usetheme{Boadilla}
%\usetheme{CambridgeUS}
%\usetheme{Copenhagen}
%\usetheme{Darmstadt}
%\usetheme{Dresden}
%usetheme{Frankfurt}
%\usetheme{Goettingen}
%\usetheme{Hannover}
%\usetheme{Ilmenau}
%\usetheme{JuanLesPins}
%\usetheme{Luebeck}
\usetheme{Madrid}
%\usetheme{Malmoe}
%\usetheme{Marburg}
%\usetheme{Montpellier}
%\usetheme{PaloAlto}
%\usetheme{Pittsburgh}
%\usetheme{Rochester}
%\usetheme{Singapore}
%\usetheme{Szeged}
%\usetheme{Warsaw}

% As well as themes, the Beamer class has a number of color themes
% for any slide theme. Uncomment each of these in turn to see how it
% changes the colors of your current slide theme.

%\usecolortheme{albatross}
%\usecolortheme{beaver}
%\usecolortheme{beetle}
%\usecolortheme{crane}
%\usecolortheme{dolphin}
%\usecolortheme{dove}
%\usecolortheme{fly}
%\usecolortheme{lily}
%\usecolortheme{orchid}
%\usecolortheme{rose}
%\usecolortheme{seagull}
%\usecolortheme{seahorse}
%\usecolortheme{whale}
%\usecolortheme{wolverine}

%\setbeamertemplate{footline} % To remove the footer line in all slides uncomment this line
%\setbeamertemplate{footline}[page number] % To replace the footer line in all slides with a simple slide count uncomment this line

%\setbeamertemplate{navigation symbols}{} % To remove the navigation symbols from the bottom of all slides uncomment this line
}

\setbeamertemplate{items}[triangle]
\setbeamertemplate{footline}{}
\addtobeamertemplate{navigation symbols}{}{%
	\insertframenumber/\inserttotalframenumber
}
\usepackage{graphicx} % Allows including images
\usepackage{booktabs} % Allows the use of \toprule, \midrule and \bottomrule in tables
\usepackage{minted}
%----------------------------------------------------------------------------------------
%	TITLE PAGE
%----------------------------------------------------------------------------------------
\title[Rcpp]{Rcpp} % The short title appears at the bottom of every slide, the full title is only on the title page

\author{Oliver Heidmann\\{\small Supervised by: Julian Kunkel}}
\institute[UNIHH] 
{
University of Hamburg  \\ % Your institution for the title page
\medskip
\textit{oliverheidmann@hotmail.de} % Your email address
\newline
}
\date{\today} % Date, can be changed to a custom date

\begin{document}
\begin{frame}
\titlepage % Print the title page as the first slide
\end{frame}

% - - - - - - - - - - - - - - - - - - - - - - - -
\begin{frame}
\frametitle{Overview} \tableofcontents 
\end{frame}

%DONE
%===============================================
\section{What is C++?}
%===============================================
\begin{frame}
    \frametitle{What is C++?}
% - - - - - - - - - - - - - - - - - - - - - - - -
\begin{itemize}
    \item object orientated programming languge
    \item intermediate level language
        \begin{itemize}
            \item high level language features
            \newline and also 
            \item low level language features 
        \end{itemize}
    \item designed to be
        \begin{itemize}
            \item fast at executing 
            \item efficient with memory
            \item flexible in the way it can be used
        \end{itemize}
\end{itemize}
% - - - - - - - - - - - - - - - - - - - - - - - -
\end{frame}

%===============================================
\section{What is Rccp?}
%===============================================
\begin{frame}
    \frametitle{What is Rccp?}
% - - - - - - - - - - - - - - - - - - - - - - - -
\setbeamertemplate{items}[triangle]
\begin{itemize}
    \item easy integration of c++ code
    \item mappings from r objects to c classes 
    \item package skeleton creation
    \item flexible error and exception handling
\end{itemize}
\end{frame}

%DONE
%DONE
%===============================================
\section{Rcpp basics}
%===============================================
\subsection{RObject and SEXP}
\begin{frame}
\frametitle{RObject and SEXP}
% - - - - - - - - - - - - - - - - - - - - - - - -
\begin{itemize}
    \item Rcpp::RObject is a very thin wrapper around an SEXP. 
    \item Rcpp::RObject defines set of functions applicable to any r object
    \item SEXP only member of Rcpp::RObject.
    \item SEXP represents r object
    \item SEXP is guarded from Garbage collection through Rcpp::RObject
\end{itemize}
% - - - - - - - - - - - - - - - - - - - - - - - -
\end{frame}        

%DONE
%================================================
\subsubsection{Rcpp basics: Type Mappings}
\begin{frame}
\frametitle{Rcpp basics: Type Mappings}
% - - - - - - - - - - - - - - - - - - - - - - - -
    What is Wrappable?
\begin{itemize}
    \item int double bool to R atomic vectors
    \item std::string to R atomic character vectors
    \item STL containers
    \item and any class that has a SEXP() operator for conversion
    \item or any class in which wrap() tamplate is specialized
\end{itemize}
% - - - - - - - - - - - - - - - - - - - - - - - -
\end{frame}

%================================================
\subsubsection{conversion methods}
\begin{frame}[fragile]
\frametitle{conversion methods}
% - - - - - - - - - - - - - - - - - - - - - - - -
Rcpp::wrap for converting c++ types to R
\begin{minted}{c++}
template <typename T> SEXP wrap(const T &  object)
\end{minted}
\hspace{4 mm} \\
Rcpp::as for converting R types to c++ types
\begin{minted}{c++}
template <typename T> T as(SEXP x)
\end{minted} 
% - - - - - - - - - - - - - - - - - - - - - - - -
\end{frame}

%================================================
\subsubsection{Calling a function}
\begin{frame}[fragile]
\frametitle{Calling a c++ function in r}
% - - - - - - - - - - - - - - - - - - - - - - - -
\begin{itemize}
    \item calling done with .call(...)
    \item r function call
    \item calls the c function for r
\end{itemize}
\begin{minted}{R}
.call(
    "function_name", parameter_1, ...,
    parameter_n, package="packagename"
)
\end{minted}
% - - - - - - - - - - - - - - - - - - - - - - - -
\end{frame}

%================================================
\section{Writing R code}
%================================================
\subsection{Rcpp.package.skeleton}
\begin{frame}
    \frametitle{Rcpp.package.skeleton}
% - - - - - - - - - - - - - - - - - - - - - - - -
\begin{itemize}
    \item r command to create skeleton rcpp package
    \item already in Rcpp package
    \item can be created with example functions
    \item very easy to get into
\end{itemize}
% - - - - - - - - - - - - - - - - - - - - - - - -
\end{frame}

%===============================================
\begin{frame}
    \frametitle{rcpp skeleton package guide}
% - - - - - - - - - - - - - - - - - - - - - - - -
    rough guide in package included
\begin{itemize}
\item edit the help file skeletons in 'man', possibly combining help files
  for multiple functions.
\item edit the exports in 'NAMESPACE', and add necessary imports.
\item put any c/c++/fortran code in 'src'.
\item R CMD build to build the package tarball.
\item R CMD check to check the package tarball.
\end{itemize}
% - - - - - - - - - - - - - - - - - - - - - - - -
\end{frame}

%===============================================
\section{Example parts}
%===============================================
\subsection{cpp Export function}
\begin{frame}[fragile]
\frametitle{cpp Export function}
% - - - - - - - - - - - - - - - - - - - - - - - -
\begin{minted}{c++}
RcppExport SEXP test_add_lists(SEXP r_list1, SEXP r_list2)
{        
    std::vector<int> cpp_vector1;
    std::vector<int> cpp_vector2;
        
    BEGIN_RCPP 
    Rcpp::RObject __result;
    Rcpp::RNGScope __rngScope;

    cpp_vector1 = Rcpp::as<std::vector<int>>(r_list1);
    cpp_vector2 = Rcpp::as<std::vector<int>>(r_list2);
    
    __result = add_lists(cpp_vector1, cpp_vector2);
    
    return Rcpp::wrap(__result);
    END_RCPP
}
\end{minted}
% - - - - - - - - - - - - - - - - - - - - - - - -
\end{frame}

%===============================================
\subsection{c++ function}
\begin{frame}[fragile]
\frametitle{c++ function}
% - - - - - - - - - - - - - - - - - - - - - - - -
\begin{minted}{c++}

    std::vector<int> add_lists(std::vector<int> vec1,
                               std::vector<int> vec2) {
        
        std::vector<int> result;
        unsigned long max_length;
        
        max_length = std::min(vec1.size(), vec2.size());
        
        for (unsigned long i = 0; i < max_length; i++)
        {
            result.push_back(vec1[i] + vec2[i]);
        }
        return result;}

\end{minted} 
% - - - - - - - - - - - - - - - - - - - - - - - -
\end{frame}

%===============================================
\subsection{calling the function}
\begin{frame}[fragile]
\frametitle{calling the function}
% - - - - - - - - - - - - - - - - - - - - - - - -

\begin{minted}{c++}
    add_lists <-function(vec1, vec2) {
       .Call( "test_add_lists", vec1, vec2, PACKAGE = 'test')
   }
\end{minted}

% - - - - - - - - - - - - - - - - - - - - - - - -
\end{frame}

%================================================
\subsection{package inline}
\begin{frame}[fragile]
\frametitle{package inline}
% - - - - - - - - - - - - - - - - - - - - - - - -
\begin{minted}{c++}
add <- cppFunction("
    double add(double x, double y)
    {
    double sum = x + y;
    return sum;
    }
")

add(-0.01,1.02)
\end{minted}
output:
1.01
% - - - - - - - - - - - - - - - - - - - - - - - -
\end{frame}

%------------------------------------------------
\subsection{sourceCpp}
\begin{frame}[fragile]
\frametitle{sourceCpp}
% - - - - - - - - - - - - - - - - - - - - - - - -
simple-c-functions.cpp
\begin{minted}{c++}
#include <Rcpp.h>
#include <iostream>

// [[Rcpp::export]]
double myCmean(Rcpp::NumericVector x)
{
   
    double result = 0;
    for(auto v : x)
    {
        result += v;
    }
    return result/x.length();
}
\end{minted}
% - - - - - - - - - - - - - - - - - - - - - - - -
\end{frame}

%------------------------------------------------
\begin{frame}[fragile]
\frametitle{sourceCpp}
% - - - - - - - - - - - - - - - - - - - - - - - -
main.R
\begin{minted}{c++}

    require(Rcpp)
    sourceCpp("simple-c-functions.cpp")

    myCmean(10.01,1.1,2.26)
\end{minted}
output: \newline
13.37

% - - - - - - - - - - - - - - - - - - - - - - - -
\end{frame}
%DONE

%===============================================
\subsection{RuntimeComparison}
\begin{frame}
\frametitle{Runtime comparison}
% - - - - - - - - - - - - - - - - - - - - - - - -

x = vector with 10000 entrys of type numeric/double\newline
each function was executed 1000 times\newline
\newline
unit: microseconds\newline

\begin{tabular}{rrrr}
function & min & median & max \\
r inlineMean(x) & 12.054 & 14.5440 & 49.531 \\
r mean(x) & 2062.821 & 2151.4090 & 3021.454 \\
c mean(x) & 9.179 & 9.8325 & 21.462 \\
\end{tabular}
% - - - - - - - - - - - - - - - - - - - - - - - -
\end{frame}

%===============================================
\section{Conclusion}
\begin{frame}
\frametitle{Conclusion}
% - - - - - - - - - - - - - - - - - - - - - - - -
\begin{itemize}
\item Rcpp is easy to get into
\item package creation is fast and easy
\item sourceCpp() easiest way of using small
      cpp functions
\item c++ code is a lot faster
    
\end{itemize}
\end{frame}

%===============================================
\section{sources}
\begin{frame}
\frametitle{sources}
% - - - - - - - - - - - - - - - - - - - - - - - -
\begin{itemize}
\item http://dirk.eddelbuettel.com/code/rcpp/Rcpp-introduction.pdf\newline13.6.2016
\item http://dirk.eddelbuettel.com/code/rcpp/Rcpp-package.pdf\newline 13.6.2016
\item http://dirk.eddelbuettel.com/code/rcpp/Rcpp-FAQ.pdf\newline13.6.2016
\item https:://www.techopedia.com/definition/26184/c-programming-language\newline13.6.2016
\item adv-r.had.co.nz/Rcpp.html\newline13.6.2016
\item https://cran.r-project.org/web/packages/microbenchmark/microbenchmark.pdf\newline13.6.2016
% - - - - - - - - - - - - - - - - - - - - - - - -
\end{itemize}
% - - - - - - - - - - - - - - - - - - - - - - - -
\end{frame}
%===============================================
\end{document} 
