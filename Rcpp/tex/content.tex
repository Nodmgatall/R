\documentclass{article}

\usepackage{minted}
\usepackage[hidelinks]{hyperref}
\hypersetup{
    colorlinks=true,
    linkcolor=blue,
    filecolor=black,
    urlcolor=blue,
}


\title{Content Presentation}

\date{23.5.2016}

\author{Oliver Heidmann}

\begin{document}
\maketitle
\newpage
\tableofcontents
\newpage

\section{What is Rcpp}

    R is for easily acessing c functions in R
    +better description
    +more facts
\section{How to setup Rcpp}

    \subsection{to use in C/C++ code}
        simply include \textless Rcpp.h\textgreater
        \newline
        install.packages('Rcpp')
        \newline
        install.packages('inline')

\section{Rcpp basics}
    \subsection{RObject and SEXP}
        Rcpp::RObject are very thin wrappers around an SEXP. 
        In fact the SEXP is the only member of the Rcpp::RObject.(check if fact or just
        misunderstanding)
        \newline
        SEXP is guarded from Garbage collection through Rcpp::RObject

        SEXP types:
            
    \subsection{conversions}
        \subsubsection{what is wrapable}
        \subsubsection{C++ to R}
            Rcpp::wrap:
            \begin{minted}{c++}
    template <typename T> SEXP wrap(const T& object)
            \end{minted}
            +expl
        \subsubsection{R to C++}
        \begin{minted}{c++}
    template <typename T> T as(SEXP x)
        \end{minted} 
        + expl
    \subsection{Structure of Rcpp code}
        better explanation of " first convert , calculate, convert
    \subsection{Calling a function}
        explanation of .call
\section{ two ways to use Rcpp}

    \subsection{Using Rcpp Inline}

    what happens with the c++ inline function when executed - explanation
        \begin{minted}{c++}
    cppFunction("double add(double x, double y, double z){
        double sum = x + y + z;
        return sum;
        }"
    )

add(2,33,24.3)        \end{minted}


    \subsection{Rcpp packages}
        Rcpp.package.skeleton - explanation
        \newline
        Example C++ function
            \begin{minted}{c++}

    std::vector<int> add_lists(std::vector<int> vec1, std::vector<int> vec2) {
        unsigned long max_length = std::min(vec1.size(), vec2.size());
        std::vector<int> result;
        for (unsigned long i = 0; i < max_length; i++) {
            result.push_back(vec1[i] + vec2[i]);
        }
        return result;}

            \end{minted}
        Example binding to R function
            \begin{minted}{c++}
    add_lists <-function(vec1, vec2) {
       .Call( "test_add_lists", vec1, vec2, PACKAGE = 'test')
   }
            \end{minted}
        Example Export Wrapper function
            \begin{minted}{c++}
    RcppExport SEXP test_add_lists(SEXP vec1, SEXP vec2) {
        BEGIN_RCPP
        Rcpp::RObject __result;
        Rcpp::RNGScope __rngScope;
        __result =
        add_lists(Rcpp::as<std::vector<int>>(vec1), Rcpp::as<std::vector<int>>(vec2));
        return Rcpp::wrap(__result);
        END_RCPP
    }
            \end{minted}

    \section{Paralell code(OPTIONAL NO IDEA IF IN FINAL VERSION OR NOT)}
        \subsection{in C++ code used in R}
        \subsection{in R}

    \section{Examples(Selfwritten and more)}
        \section{self}
            
        \section{other}
    \section{Source}
    \url{https://stat.ethz.ch/R-manual/R-devel/library/parallel/doc/parallel.pdf}
        \newline
        \url{http://dirk.eddelbuettel.com/code/rcpp/Rcpp-introduction.pdf}
        \newline
    \url{http://dirk.eddelbuettel.com/code/rcpp/Rcpp-package.pdf}
        \newline
    \url{http://dirk.eddelbuettel.com/code/rcpp/Rcpp-FAQ.pdf}
        \newline

REAL CITATIONS AS ARE WISHED FOR IN THOSE PDF WILL FOLLOW!!!
\end{document}
